\documentclass[totpages,helvetica,openbib,italian]{europecv}
\usepackage[T1]{fontenc}
\usepackage{graphicx}
\usepackage[a4paper,top=1.27cm,left=1cm,right=1cm,bottom=2cm]{geometry}
\usepackage[italian]{babel}
\usepackage{bibentry}
\usepackage{url}
\usepackage{lipsum,environ}

\newsavebox\recipebox
\newlength\recipebreaklength
\setlength\recipebreaklength{0.4\textheight}
\newcommand\typesetrecipe[1]{%
  \par\noindent\rule{\linewidth}{1ex}\par
  \BODY
  \par\noindent\rule{\linewidth}{1ex}\par
}
\NewEnviron{recipe}{%
  \savebox{\recipebox}{\parbox{\linewidth}{%
    \typesetrecipe{\BODY}%
  }}%
  \ifdim
      \dimexpr\ht\recipebox+\dp\recipebox\relax 
      > \recipebreaklength
    \typesetrecipe{\BODY}%
  \else
    \par\noindent\usebox\recipebox
  \fi
  \bigskip
}
\renewcommand{\ttdefault}{phv} % Uses Helvetica instead of fixed width font

\ecvname{Bencivenga Enrico}
%\ecvfootername{Nome/i Cognome/i}
\ecvaddress{Via Cucca 106, 80031 Brusciano (Napoli)}
\ecvtelephone[]{+393289196064}
%\ecvfax{Facoltativo}
\ecvemail{\url{enrico.bencivenga@hotmail.it}}
\ecvnationality{Italiana}
\ecvdateofbirth{13/04/1983}
\ecvgender{Maschile}
%\ecvpicture[width=3cm]{mypicture}
\ecvfootnote{Per ulteriori informazioni: \url{http://europass.cedefop.eu.int}\\
\textcopyright~European Communities, 2003.}

\begin{document}
\selectlanguage{italian}


\begin{europecv}
\ecvpersonalinfo[20pt]


\ecvsection{Istruzione e formazione}

\ecvitem{Date}{Aprile 2012 - Aprile 2013}
\ecvitem{Nome e tipo d'istituto di istruzione o formazione}{Ancitel S.p.A., Via Vicinale Santa Maria del Pianto 80143 Napoli }
\ecvitem{Certificato o diploma ottenuto}{Tecnico di ricerca sui servizi della PA. Progetto SMART (Services and Meta-services for smART eGovernment)}
\ecvitem{Principali materie/Competenze professionali apprese}{\begin{minipage}[t]{1\linewidth}
Strumenti di sviluppo di architetture SOA e testing :
\begin{itemize}
\vspace{-3mm}\item[]    Scrittura del codice: Eclipse
\vspace{-3mm}\item[]    Gestione codice sorgente: SVN e GIT
\vspace{-3mm}\item[]    Gestione delle issue: Mantis, Bugzilla e Request Tracker
\vspace{-3mm}\item[]    Supporto allo sviluppo: JUnit, DBUnit
\vspace{-3mm}\item[]    Testing: Hudson, Selenium, HTTPerf
\vspace{-3mm}\item[]    Linguaggi XML e XSD
\vspace{-3mm}\item[]    Formalismo WSDL
\vspace{-3mm}\item[]    Protocollo SOAP
\vspace{-3mm}\item[]    Web services: progettazione, implementazione, consumo
\vspace{-3mm}\item[]    Sistemi RESTful
    \end{itemize}

Metodologie di sviluppo Agile : Scrum ed XP.\\
\vspace{1mm}

Identità Digitale :
\begin{itemize}

\vspace{-3mm}\item[]   Sistemi a directory: LDAP, Active Directory;
\vspace{-3mm}\item[]   Single Sign On in ambito intranet ed extranet: OATH, OpenID, SAML 2.0;
\vspace{-3mm}\item[]   Sistemi di autorizzazione e di delega: XACML, OAuth 2.0;
\vspace{-3mm}\item[]   Federazioni di identità in Italia: People/ICAR, FedERa.
\end{itemize}

Cloud computing:
\begin{itemize}
   \vspace{-3mm}\item[]Architetture.
   \vspace{-3mm}\item[]Dati: sistemi data-intensive (come Apache Hadoop).
   \vspace{-3mm}\item[]Applicazioni : Google App Engine, Microsoft Azure, Red Hat OpenShift.
   \end{itemize}

Modellazione:
\begin{itemize}

 \vspace{-3mm}\item[]   Requisiti software e casi d’uso, Modello Entità-Relazione, Linguaggio UML,
\vspace{-3mm}\item[]    Modello dati relazionale
    \end{itemize}
Redazione di documentazione tecnica: Doxygen.\\
\vspace{1mm}

Programmazione Java e ad oggetti.\\
\vspace{1mm}

Governo elettronico e CAD (Codice dell’Amministrazione Digitale):
\begin{itemize}

\vspace{-3mm}\item[] Servizi e procedimenti della PA (Centrale e Locale).
\vspace{-3mm}\item[]    Struttura della PA (Centrale e Locale).
\end{itemize}
  Sistemi territoriali:
  \begin{itemize}

  \vspace{-3mm}\item[]  Strumenti GIS per la catalogazione e gestione dei dati
\vspace{-3mm}\item[]    Strumenti per uniformare i dati territoriali
    ISPIRE ( metadati geografici)
\vspace{-3mm}\item[]    Standard OGC ( per la condivisione dei dati)
    Geonetwork (catalogo metadati)
\vspace{-3mm}\item[]    Strumento "Repertorio Nazionale dei Dati Territoriali"
\vspace{-3mm}\item[]    Web Server (Apache, linguaggio PHP)
\vspace{-3mm}\item[]    Map Server
\vspace{-3mm}\item[]    Geowebcache
\vspace{-3mm}\item[]    GeoServer
\vspace{-3mm}\item[]    API (Google) e interfacce di accesso (Visualizzatori web GIS).
    \end{itemize}

Semantic Web.
\vspace{1mm}
\vspace{1mm}

iPhone Developer:
\begin{itemize}

\vspace{-3mm}\item[]    Conoscenza dei dispositivi iPhone, iPod Touch e iPad,
\vspace{-3mm}\item[]    iOS technology layers: Core OS, Core Services, Media, Cocoa Touch
 \vspace{-3mm}\item[]   Strumenti di sviluppo per la piattaforma iOS: Xcode, Objective C.
    \end{itemize}
Sviluppo applicazioni Android con Eclipse\\
\vspace{1mm}

Programmazione e gestione operativa dei Progetti di ricerca.\\
\end{minipage}
}


\ecvitem{Date}{Settembre 2010 - Giugno 2011}
\ecvitem{Nome e tipo d'istituto di istruzione o formazione}{
\begin{minipage}[t]{1\linewidth}
Università di Paderborn (Germania)\\
 Faculty of Computer Science and Mathematics
 \end{minipage}
 }
\ecvitem{Principali materie}{
\vspace{-3mm}
\begin{itemize}
\vspace{-3mm}\item[]Model Driven Software Development;
\vspace{-3mm}\item[] Web Engineering;
\vspace{-3mm}\item[]Deductive Verification;
\vspace{-3mm}\item[] Modelchecking;
\vspace{-3mm}\item[] Swarm Intelligence.
\end{itemize}}
%\ecvitem{Livello nella classificazione nazionale o internazionale\footnote{Se pertinente.}}{\ldots}

\ecvitem{Date}{2002 - oggi}
\ecvitem{Nome e tipo d'istituto di istruzione o formazione}{
\begin{minipage}[t]{1\linewidth}
Università degli Studi di Napoli "Federico II"\\ Corso di Laurea in Ingegneria Informatica 
\end{minipage}
}
\ecvitem{Certificato o diploma da ottenere}{Laurea in Ingegneria Informatica}
\ecvitem{Principali materie}{
\vspace{-3mm}
\begin{itemize}
\vspace{-3mm}\item[]Ingegneria del Software;
\vspace{-3mm}\item[] Programmazione I;
\vspace{-3mm}\item[] Programmazione II; 
\vspace{-3mm}\item[]Tecnologie dei Sistemi di Automazione. \end{itemize}}

\ecvsection{Capacit\`a e competenze professionali}

\ecvmothertongue[30pt]{Italiano}
\ecvitem{\large Altre lingue}{}
\ecvlanguageheader{(*)}
\ecvlanguage{Inglese}{\ecvCOne}{\ecvCOne}{\ecvCOne}{\ecvCOne}{\ecvCOne}
\ecvlastlanguage{Tedesco}{\ecvBOne}{\ecvBOne}{\ecvBOne}{\ecvBOne}{\ecvBOne}
\ecvlanguagefooter[10pt]{(*)}

\ecvitem[10pt]{\large Capacit\`a e competenze sociali}{Ottime capacità sociali, sviluppate fin da piccolo, grazie alle attività sportive
praticate (calcio, podismo, ciclismo) ed, in seguito, maturate durante il periodo
Erasmus in Germania, in cui ho avuto l'occasione di relazionarmi con persone di diversi paesi e culture.}
\ecvitem[10pt]{\large Capacit\`a e competenze organizzative}{Ottime capacità organizzative, maturate durante gli anni in cui ho allenato i
ragazzi delle categorie Pulcini ed Esordienti a Brusciano (NA), in cui mi sono
confrontato con la gestione del gruppo, dei rapporti con la società di
appartenenza e con la Federazione.
Per quanto riguarda l'Università, credo di aver avuto una buona maturazione
durante i lavori di gruppo svolti per alcuni esami ed, infine, durante il periodo
Erasmus, in cui ho fatto parte del progetto Safebots.}

%\ecvitem[10pt]{\large Capacit\`a e competenze tecniche}{Descrivere tali competenze e indicare dove sono state acquisite. Facoltativo.}

\ecvitem[10pt]{\large Capacit\`a e competenze informatiche}{\begin{minipage}[t]{1\linewidth}
Ottima conoscenza dei sistemi operativi:
\begin{itemize}
\vspace{-3mm}\item[] Microsoft: Windows 98, ME, 2000, XP, 7;
\vspace{-3mm}\item[] Apple: MacOS X Snow Leopard, Lion, Mountain Lion, Mavericks;
\vspace{-3mm}\item[] Linux: Archlinux, RedHat, SuSE.
\end{itemize}

Ottima conoscenza delle piattaforme office:
\begin{itemize}
\vspace{-3mm}\item[]Microsoft Office 2003, 2007, 2010, 2013;
\vspace{-3mm}\item[]OpenOffice;
\vspace{-3mm}\item[]LibreOffice.
\end{itemize}

Ottima conoscenza dei browsers:
\begin{itemize}
\vspace{-3mm}\item[] Internet Explorer;
\vspace{-3mm}\item[] Mozilla Firefox;
\vspace{-3mm}\item[] Safari;
\vspace{-3mm}\item[] Google Chrome.
\end{itemize}

Ottima conoscenza dei clients di posta elettronica:
\begin{itemize}
\vspace{-3mm}\item[]Outlook Express;
\vspace{-3mm}\item[]Mozilla Thunderbird.
\end{itemize}

Ottima conoscenza dei software grafici:
\begin{itemize}
\vspace{-3mm}\item[] Adobe Photoshop;
\vspace{-3mm}\item[]The Gimp.

\end{itemize}

Ottima conoscenza del software per la grafica vettoriale InkScape.\\
\vspace{1mm}

Ottima conoscenza dei software per il calcolo matematico:
\begin{itemize}
\vspace{-3mm}\item[]Matlab;
\vspace{-3mm}\item[]Derive.
\end{itemize}


Ottima conoscenza delle piattaforme di sviluppo:
\begin{itemize}
\vspace{-3mm}\item[]Microsoft Visual Studio;
\vspace{-3mm}\item[]Eclipse;
\vspace{-3mm}\item[]NetBeans;
\vspace{-3mm}\item[]XCode;
\vspace{-3mm}\item[]Kdevelop;
\vspace{-3mm}\item[]Monodevelop.
\end{itemize}

Ottima conoscenza dei linguaggi di programmazione:
\begin{itemize}
\vspace{-3mm}\item[]C\#;
\vspace{-3mm}\item[]C++;
\vspace{-3mm}\item[]Objective-C;
\vspace{-3mm}\item[]Java;
\vspace{-3mm}\item[]Php.
\end{itemize}

Buona conoscenza dei linguaggi di scripting:
\begin{itemize}
\vspace{-3mm}\item[]Python;
\vspace{-3mm}\item[]Perl;
\vspace{-3mm}\item[]Ruby.
\end{itemize}

Ottima conoscenza dei DBMS relazionali:
\begin{itemize}
\vspace{-3mm}\item[]Microsoft SQL Server 2005, 2008, 2010, 2012;
\vspace{-3mm}\item[]MySQL 5.5, 5.6, 5.7, 6.0.
\end{itemize}

Ottima conoscenza del linguaggio di modellazione UML.\\
\newline
Ottima conoscenza del software di composizione tipografica Tex e del linguaggio di markup LaTex.

\end{minipage}
}
\ecvitem[10pt]{\large Capacit\`a e competenze artistiche}{Ho studiato musica per alcuni anni suonando il clarinetto e, da autodidatta, il
pianoforte.}
%\ecvitem[10pt]{\large Altre capacit\`a e competenze}{Descrivere tali competenze e indicare dove sono state acquisite. Facoltativo.}
\ecvitem{\large Patente}{Patente B}

%\textsc{\ecvsection{Ulteriori informazioni}
%\ecvitem[10pt]{}{Inserire qui ogni altra informazione utile, ad esempio persone di riferimento, referenze, etc\ldots Facoltativo.}
%\bibliographystyle{plain}
%\nobibliography{publications}
%\ecvitem{}{\textbf{Pubblicazioni}}
%\ecvitem{}{\bibentry{pub1}}
%\ecvitem[10pt]{}{\bibentry{pub2}}}
\newline
\vspace{10mm}
\ecvitem{}{
\begin{minipage}[t]{1\linewidth}
\textbf{Interessi personali}\\
Lettura; pratica del ciclismo e del podismo.
\end{minipage}
}
\ecvitem{Firma}{\small \emph{Ai sensi del D. Lgs. 196 del 30/6/2003, autorizzo al trattamento dei miei dati personali per le vostre esigenze di selezione e comunicazione e dichiaro di essere informato dei diritti di cui l'articolo 13 a me spettanti} \par }

\end{europecv}


\end{document} 